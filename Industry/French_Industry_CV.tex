\documentclass[12pt,a4paper,roman]{moderncv} % Font sizes: 10, 11, or 12; paper sizes: a4paper, letterpaper, a5paper, legalpaper, executivepaper or landscape; font families: sans or roman

\usepackage[main=french, english, italian]{babel}
 \frenchbsetup{StandardLists=true}

\moderncvcolor{burgundy} % CV color - options include: 'blue' (default), 'orange', 'green', 'red', 'purple', 'grey' and 'black'
\moderncvstyle{classic} % CV theme - options include: 'casual' (default), 'classic', 'oldstyle' and 'banking'

\usepackage[scale=0.9]{geometry} % Reduce document margins
\setlength{\makecvheadnamewidth}{8cm} % For the 'classic' style, uncomment to adjust the width of the space allocated to your name
\usepackage{comment}
\usepackage{hyperref}
\usepackage{eurosym}

\hypersetup{bookmarksnumbered=true, colorlinks=true}


%----------------------------------------------------------------------------------------
%	NAME AND CONTACT INFORMATION SECTION
%----------------------------------------------------------------------------------------


\name{Carlo}{Buccisano}

% All information in this block is optional, comment out any lines you don't need
\title{Curriculum Vitae}
\email{cbuccisano.work@pm.me}
\homepage{carlobuccisano.github.io}
%\phone{}
\social[github]{carlobuccisano}
\social[linkedin]{carlobuccisano}
\extrainfo{Langues : italien, anglais, français et un peu d'espagnol}
%----------------------------------------------------------------------------------------

\begin{document}

\makecvtitle % Print the CV title

Doctorat en mathématiques pures à la recherche d'un poste en Data Science. 


%----------------------------------------------------------------------------------------
%	WORK EXPERIENCE SECTION
%----------------------------------------------------------------------------------------
\section{Expérience Professionnelle}
\cventry{2024--2025}{ATER en Mathématiques}{Université Paris-Saclay (Orsay)}{}{}{
	\textcolor{darkgray}{Enseignement de 192 heures de cours de mathématiques fondamentales en français en licence.	\\ Compétences relationnelles : communication et leadership.}
}

\cventry{2021--2025}{PhD en Mathématiques}{Université Paris-Saclay (Orsay)}{}{}{
	\begin{itemize}
		\item \textcolor{darkgray}{ J'ai mené des recherches en géométrie algébrique dérivée, un mélange entre la géométrie algébrique et la théorie de l'homotopie, où j'ai utilisé les D-modules dérivés pour relier plusieurs concepts dans la littérature. Ma thèse est disponible sur \href{https://theses.hal.science/tel-05351522v1}{HAL} et sur \href{https://arxiv.org/abs/2510.15665}{arXiv}. 
		\item J'ai co-organisé la conférence GAeL XXIX, qui a réuni plus de 100 personnes à Orsay.
	\item J'ai donné plusieurs conférences publiques et présenté des posters à des publics d'horizons divers.}
	\end{itemize}
	\textcolor{darkgray}{Compétences relationnelles : gestion du temps, communication et problem solving.}
}

\cventry{Juillet 2015}{Programmeur Informatique}{IBM}{UK}{}{
	\begin{itemize}
		\item \textcolor{darkgray}{ J'ai travaillé en bin{\^o}me avec un autre collègue, au sein d'une équipe de 5 personnes.
		 \item J'ai développé et documenté certaines API d'un outil de planification interne en Python.  
		\item Prix de la Banque d'Italie pour l'étude des mathématiques et de l'informatique.  }
\end{itemize}
	\textcolor{darkgray}{Compétences relationnelles : travail d'équipe et gestion de projet.}
}

\section{Compétences Techniques}

%\cvitem{}{ {\small \begin{itemize}
%	\item \textcolor{darkgray}{J'ai appris HTML, CSS et Javascript en codant ma page web personelle.
%\item Je sais coder en \texttt{C++} et \texttt{Python} (grâce à les competitions et à des cours de licence).
%	\item J'ai appris et maîtrisé \LaTeX\, pendant mon doctorat, pour rédiger ma thèse.
%\item  J'utilise Arch Linux au quotidien : je maîtrise bien les scripts Bash et les conteneurs Docker.
%\item  J'utilise le contrôle de version \texttt{git}, à la fois depuis le terminal et depuis l'interface web GitHub.}
%\end{itemize} } }

\cvitem{Familier}{HTML, CSS, JavaScript (j'ai codé mon site web)}
\cvitem{}{\texttt{git}, Docker (j'héberge des logiciels sur mon serveur)}
\cvitem{Compétent}{\texttt{C++}, \texttt{Python} (programmation compétitive), Script bash sous Linux, \LaTeX}



%----------------------------------------------------------------------------------------
%	EDUCATION SECTION
%----------------------------------------------------------------------------------------

\section{Éducation}
\cventry{2021--2025}{PhD en Mathématiques}{Université Paris-Saclay (Orsay)}{}{}{}
\cventry{2016--2022}{Dipl{\^o}me Galiléen}{Scuola Galileiana di Studi Superiori, Padova (Italy)}{}{}{
	\textcolor{darkgray}{
		Diplômé de la \href{https://scuolagalileiana.unipd.it/}{Scuola Galileiana}, un institut universitaire d'excellence à Padoue (taux d'admission de 7\%). Cours pertinents : Machine Learning et programmation concurrente (Java).
}
}

\cventry{2019--2021}{Master en Mathématiques}{Università degli Studi di Padova \& Université Paris-Saclay}{}{}{\textcolor{darkgray}{
		Diplômé du \href{https://algant.eu/}{double diplôme Algant} avec mention \textit{Très Bien}, première année à Padova (30/30) puis à Orsay (18.7/20). Bénéficiaire d'une bourse de 10 000\,\euro\, de la FMJH.
}}
\cventry{2016--2019}{Licence en Mathématiques}{Università degli Studi di Padova}{}{}
{\textcolor{darkgray}{Moyenne générale : 29.965/30; \textit{110/110 cum laude} (équivalent à ``Très bien'').}}


%----------------------------------------------------------------------------------------
%	AWARDS SECTION
%----------------------------------------------------------------------------------------
\section{Prix}

\cventry{2013--2015}{Médailles aux Olympiades italiennes de mathématiques et d'informatiques}{}{}{}{
 \textcolor{darkgray}{
	 J'ai obtenu deux médailles d'or (une troisième et une première place absolues) et une médaille de bronze aux Olympiades italiennes d'informatique, ainsi qu'une médaille d'argent aux Olympiades italiennes de mathématiques.\\ Compétences : résolution de problèmes, travail sous pression, structures de données et algorithmes.}}

\cventry{2013--2015}{Programmation compétitive}{}{}{}{ \textcolor{darkgray}{
Participation à plusieurs concours en ligne, tels que Google Code Jam, HashCode, Codeforces.}}

\end{document}

