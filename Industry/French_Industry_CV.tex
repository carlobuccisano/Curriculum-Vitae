\documentclass[12pt,a4paper,roman]{moderncv} % Font sizes: 10, 11, or 12; paper sizes: a4paper, letterpaper, a5paper, legalpaper, executivepaper or landscape; font families: sans or roman

\usepackage[main=french, english, italian]{babel}
 \frenchbsetup{StandardLists=true}

\moderncvcolor{burgundy} % CV color - options include: 'blue' (default), 'orange', 'green', 'red', 'purple', 'grey' and 'black'
\moderncvstyle{classic} % CV theme - options include: 'casual' (default), 'classic', 'oldstyle' and 'banking'

\usepackage[scale=0.9]{geometry} % Reduce document margins
\setlength{\makecvheadnamewidth}{8cm} % For the 'classic' style, uncomment to adjust the width of the space allocated to your name
\usepackage{comment}
\usepackage{hyperref}
\usepackage{eurosym}

\hypersetup{bookmarksnumbered=true, colorlinks=true}


%%% My commands %%%

% Print soft skills resume:

\newboolean{skills}
\setboolean{skills}{false}
%\setboolean{skills}{true}

% Role to emphasize:
\newcommand{\Role}{Data Science}

% Text colors
\newcommand{\Scuro}[1]{\textcolor{darkgray}{#1}}
\newcommand{\Chiaro}[1]{\textcolor{gray}{#1}}

%----------------------------------------------------------------------------------------
%	NAME AND CONTACT INFORMATION SECTION
%----------------------------------------------------------------------------------------


\name{Carlo}{Buccisano}

% All information in this block is optional, comment out any lines you don't need
\title{Curriculum Vitae}
\email{work@carlobuccisano.com}
\homepage{carlobuccisano.com}
%\phone{}
\social[github]{carlobuccisano}
\social[linkedin]{carlobuccisano}
\extrainfo{Langues : italien, anglais, français et espagnol (A2)}
%----------------------------------------------------------------------------------------

\begin{document}

\makecvtitle % Print the CV title

Doctorat en mathématiques, avec une expérience de recherche en géométrie algébrique et en topologie algébrique, passionné par la programmation et Linux. À la recherche d'un poste en \Role.

%----------------------------------------------------------------------------------------
%	WORK EXPERIENCE SECTION
%----------------------------------------------------------------------------------------
\section{Expérience Professionnelle}
\cventry{2024--2025}{ATER en Mathématiques}{Université Paris-Saclay (Orsay)}{}{}{
	\Scuro{Enseignement de 192 heures de cours de mathématiques fondamentales en français en licence.}
	\ifthenelse{\boolean{skills}}{\\ \Scuro{Compétences relationnelles : communication et leadership.}}{\empty}
}

\cventry{2021--2025}{PhD en Mathématiques}{Université Paris-Saclay (Orsay)}{}{}{
	\begin{itemize}
		\ifthenelse{\boolean{skills}}{
			\item \Scuro{ J'ai mené des recherches en géométrie algébrique dérivée où j'ai utilisé les D-modules dérivés pour relier plusieurs concepts dans la littérature. Ma thèse est disponible sur \href{https://theses.hal.science/tel-05351522v1}{HAL} et sur \href{https://arxiv.org/abs/2510.15665}{arXiv}.}
		}{
		\item \Scuro{ J'ai mené des recherches en géométrie algébrique dérivée, un mélange entre la géométrie algébrique et la théorie de l'homotopie, où j'ai utilisé les D-modules dérivés pour relier plusieurs concepts dans la littérature et résoudre des conjectures sur la cohomologie des algebro{\"i}des de Lie. Ma thèse est disponible sur \href{https://theses.hal.science/tel-05351522v1}{HAL} et sur \href{https://arxiv.org/abs/2510.15665}{arXiv}.}
	   }
		\item \Scuro{J'ai co-organisé la conférence GAeL XXIX, qui a réuni plus de 100 personnes à Orsay.
	\item J'ai donné plusieurs conférences publiques et présenté des posters à des publics d'horizons divers.}
	\end{itemize}
	\ifthenelse{\boolean{skills}}{ \Scuro{Compétences relationnelles : gestion du temps, communication et problem solving.}}{\empty}
}

\cventry{Juillet 2015}{Programmeur Informatique}{IBM}{UK}{}{
	\begin{itemize}
		\ifthenelse{\boolean{skills}}{
			\item \Scuro{Programmation en bin{\^o}me au sein d'une équipe de 5 personnes (Agile).}
			\item \Scuro{Backend Python pour gérer des bases de données MySQL.}
		}{
		\item \Scuro{Realisation d'un outil de planification interne, programmation en bin{\^o}me au sein d'une équipe de 5 personnes (Agile).}
		\item \Scuro{Backend Python pour gérer des bases de données MySQL à partir de requêtes JSON provenant d'une application web.}
		}
		\item \Scuro{Prix de la Banque d'Italie pour l'étude des mathématiques et de l'informatique.} 
	\end{itemize}
	\ifthenelse{\boolean{skills}}{\Scuro{Compétences relationnelles : travail d'équipe et gestion de projet.}}{\empty}
}

\section{Compétences Techniques}

\cvitem{Familier}{HTML, CSS, JavaScript \Chiaro{(j'ai codé mon site web)}, SQL \Chiaro{(Stage)}}
\cvitem{}{\texttt{git}, Docker \Chiaro{(self-hosting sur mon serveur)}}
\cvitem{Compétent}{\texttt{C++} \Chiaro{(Olympiades informatique)}, \texttt{Python} \Chiaro{(stage et université)}, Script Bash, \LaTeX}



%----------------------------------------------------------------------------------------
%	EDUCATION SECTION
%----------------------------------------------------------------------------------------

\section{Éducation}
\cventry{2021--2025}{PhD en Mathématiques}{Université Paris-Saclay (Orsay)}{}{}{}
\cventry{2016--2022}{Dipl{\^o}me Galiléen}{Scuola Galileiana di Studi Superiori, Padova (Italy)}{}{}{
	\Scuro{
		Diplômé de la \href{https://scuolagalileiana.unipd.it/}{Scuola Galileiana}, un institut universitaire d'excellence à Padoue (taux d'admission de 7\%). Cours pertinents : Machine Learning et programmation concurrente (Java).
}
}

\cventry{2019--2021}{Master en Mathématiques}{Università degli Studi di Padova \& Université Paris-Saclay}{}{}{\Scuro{
		Diplômé du \href{https://algant.eu/}{double diplôme Algant} avec mention \textit{Très Bien}, première année à Padova (30/30) puis à Orsay (18.7/20). Bénéficiaire d'une bourse de 10 000\,\euro\, de la FMJH.
}}
\cventry{2016--2019}{Licence en Mathématiques}{Università degli Studi di Padova}{}{}
{\Scuro{Moyenne générale : 29.965/30; \textit{110/110 cum laude} (équivalent à ``Très bien'').}}


%----------------------------------------------------------------------------------------
%	AWARDS SECTION
%----------------------------------------------------------------------------------------
\section{Prix}

\cventry{2013--2015}{Médailles aux Olympiades italiennes de mathématiques et d'informatiques}{}{}{}{
 \Scuro{
J'ai obtenu deux médailles d'or (une troisième et une première place absolues) et une médaille de bronze aux Olympiades italiennes d'informatique, ainsi qu'une médaille d'argent aux Olympiades italiennes de mathématiques.}
\ifthenelse{\boolean{skills}}{\\ \Scuro{Compétences : résolution de problèmes, travail sous pression, structures de données et algorithmes.}}{\empty}}

\cventry{2013--2015}{Programmation compétitive}{}{}{}{ \Scuro{
Participation à plusieurs concours en ligne, tels que Google Code Jam, HashCode, Codeforces.}}

\end{document}

