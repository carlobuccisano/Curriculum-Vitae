\documentclass[12pt,a4paper,roman]{moderncv} % Font sizes: 10, 11, or 12; paper sizes: a4paper, letterpaper, a5paper, legalpaper, executivepaper or landscape; font families: sans or roman


\moderncvcolor{burgundy} % CV color - options include: 'blue' (default), 'orange', 'green', 'red', 'purple', 'grey' and 'black'
\moderncvstyle{classic} % CV theme - options include: 'casual' (default), 'classic', 'oldstyle' and 'banking'

\usepackage[scale=0.9]{geometry} % Reduce document margins
\setlength{\makecvheadnamewidth}{8cm} % For the 'classic' style, uncomment to adjust the width of the space allocated to your name
\usepackage{comment}
\usepackage{hyperref}
\usepackage{eurosym}
\usepackage[main=english, french, italian]{babel}

\hypersetup{colorlinks=true}


%----------------------------------------------------------------------------------------
%	NAME AND CONTACT INFORMATION SECTION
%----------------------------------------------------------------------------------------


\name{Carlo}{Buccisano}

% All information in this block is optional, comment out any lines you don't need
\title{Curriculum Vitae}
\email{cbuccisano.work@pm.me}
\homepage{carlobuccisano.github.io}
%\phone{}
\social[github]{carlobuccisano}
\social[linkedin]{carlobuccisano}
\extrainfo{Italian, English, French and (some) Spanish}
%----------------------------------------------------------------------------------------

\begin{document}

\makecvtitle % Print the CV title

PhD in pure mathematics looking for Data Science industry roles. 

%----------------------------------------------------------------------------------------
%	WORK EXPERIENCE SECTION
%----------------------------------------------------------------------------------------
\section{Work Experience}
\cventry{2024--2025}{Research and Teaching Assistant (ATER)}{Université Paris-Saclay (Orsay)}{}{}{
	\textcolor{darkgray}{Teaching 192 hours of undergraduate courses in French: analysis, linear algebra and geometry. \\ Soft skills: communication and leadership. 
		}
}
\cventry{2021--2025}{Ph.D. in Mathematics}{Université Paris-Saclay (Orsay)}{}{}{
		 \begin{itemize}
			 \item \textcolor{darkgray}{I led research in derived algebraic geometry, a subject mixing algebraic geometry with modern homotopy theory techniques, where I used derived D-modules to connect several concepts in the literature. My thesis can be found on \href{https://theses.hal.science/tel-05351522v1}{HAL} and on \href{https://arxiv.org/abs/2510.15665}{arXiv}. 
			\item I co-organized the GAeL XXIX conference, that brought more than 100 people to Orsay.
			\item I gave several public research talks and poster presentations to audiences of different backgrounds.}
\end{itemize}
\textcolor{darkgray}{Soft skills: time management, communication and problem solving.}
}


\cventry{July 2015}{Computer Programmer}{IBM}{UK}{}{
	\begin{itemize}
		\item \textcolor{darkgray}{I worked paired with another colleague, part of a team of 5.
		\item I developed and documented some APIs of an internal planner tool in Python.
		\item Prize by the Bank of Italy for the study of Maths and Computer Science. }
\end{itemize}
\textcolor{darkgray}{Soft skills: teamwork and project management. }}

\section{Technical Skills}

%\cvitem{}{{
	%\small \begin{itemize}
	%\item \textcolor{darkgray}{I coded my personal webpage using Hugo, for which I learnt HTML, CSS and Javascript.
	%\item I can code in \texttt{C++} and \texttt{Python} (from competitive programming and undergraduate courses).
	%\item I learnt and mastered \LaTeX\, during my PhD, to write my thesis.
	%\item  I use Arch Linux as a daily driver: I am well versed in Bash scripting and Docker containers.
%\item  I use \texttt{git} version control, both from the terminal and from GitHub web interface.}
%\end{itemize}}}
\cvitem{Familiar}{HTML, CSS, JavaScript (I coded my webpage)}
\cvitem{}{\texttt{git}, Docker (I self-host several apps on my server)}
\cvitem{Proficient}{\texttt{C++}, \texttt{Python} (competitive programming), Linux Bash scripting, \LaTeX}



%----------------------------------------------------------------------------------------
%	EDUCATION SECTION
%----------------------------------------------------------------------------------------

\section{Education}
\cventry{2021--2025}{Ph.D. in Mathematics}{Université Paris-Saclay (Orsay)}{}{}{}
\cventry{2016--2022}{Galileian Diploma}{Scuola Galileiana di Studi Superiori, Padova (Italy)}{}{}{
	\textcolor{darkgray}{Graduated from \href{https://scuolagalileiana.unipd.it/}{Scuola Galileiana}, an honors college in Padova (7\% admission rate).\\ Relevant courses: Machine Learning, Concurrent Java programming and game theory.
}
}
\cventry{2019--2021}{M.Sc. in Mathematics}{Università degli Studi di Padova \& Université Paris-Saclay}{}{}{\textcolor{darkgray}{Graduated with maximum honors from \href{https://algant.eu/}{Algant double degree}, first year in Padova (30/30) then in Orsay (18.7/20). I was awarded a 10.000\,\euro\, scholarship from FMJH.
}}
\cventry{2016--2019}{B.Sc. in Mathematics}{Università degli Studi di Padova}{}{}
{\textcolor{darkgray}{GPA: 29.965/30; \textit{110/110 cum laude} (maximum honors).}}


%----------------------------------------------------------------------------------------
%	AWARDS SECTION
%----------------------------------------------------------------------------------------
\section{Awards}
\cventry{2013--2015}{Medals in Italian Maths and Informatics Olympiads}{}{}{}{
 \textcolor{darkgray}{I obtained two gold (an absolute third and first place) and one bronze medals in Italian Informatics Olympiads, and one silver medal in Italian Mathematical Olympiads.\\ Skills: work under pressure, data structures and algorithms.
}}

\cventry{2013--2015}{Competitive Programming}{}{}{}{ \textcolor{darkgray}{Participated in online competitions such as Google Code Jam, HashCode, Codeforces, USACO.}}
\end{document}
