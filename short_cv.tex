\documentclass[10pt,a4paper,sans]{moderncv} % Font sizes: 10, 11, or 12; paper sizes: a4paper, letterpaper, a5paper, legalpaper, executivepaper or landscape; font families: sans or roman

\moderncvstyle{classic} % CV theme - options include: 'casual' (default), 'classic', 'oldstyle' and 'banking'
\moderncvcolor{grey} % CV color - options include: 'blue' (default), 'orange', 'green', 'red', 'purple', 'grey' and 'black'

\usepackage[scale=0.93]{geometry} % Reduce document margins
\setlength{\makecvheadnamewidth}{8cm} % For the 'classic' style, uncomment to adjust the width of the space allocated to your name
\usepackage{comment}
\usepackage{hyperref}

\hypersetup{bookmarksnumbered=true, colorlinks=true}


%----------------------------------------------------------------------------------------
%	NAME AND CONTACT INFORMATION SECTION
%----------------------------------------------------------------------------------------


\name{Carlo}{Buccisano}

% All information in this block is optional, comment out any lines you don't need
\title{Curriculum Vitae}
\address{Université Paris-Saclay}{Département de Mathématiques d'Orsay}{Bat 307, rue Michel Magat, Orsay, France}
\email{carlo.buccisano@universite-paris-saclay.fr}
\homepage{carlobuccisano.github.io} 
%----------------------------------------------------------------------------------------

\begin{document}

\makecvtitle % Print the CV title

%----------------------------------------------------------------------------------------
%	EDUCATION SECTION
%----------------------------------------------------------------------------------------

\section{Education}
\cventry{2024--2025}{Research and Teaching Assistant (ATER)}{Université Paris-Saclay}{Institut de Mathématiques d'Orsay}{}{Teaching various exercise sessions to undergraduate students.}
\cventry{2021--present}{PhD student in Mathematics}{Université Paris-Saclay}{Institut de Mathématiques d'Orsay}{}{Subject: D-Modules in Derived Algebraic Geometry. Advisors: Benjamin Hennion and Emanuele Macrì. }
\cventry{2020--2021}{Second year master student in Mathematics}{Université Paris-Saclay}{M2 Arithmétique, Analyse,  Géométrie}{}{GPA: 18.675/20. Mention: \textit{Très Bien} (maximum honours).}
\cventry{2019--2020}{First year master student in Mathematics}{Università degli Studi di Padova}{}{}{GPA: 30/30. }
\cventry{2016--2019}{Bachelor in Mathematics}{Università degli Studi di Padova}{}{}
	{GPA: 29.965/30. Mention: \textit{110/110 cum laude} (maximum honours).}


\section{Papers in preparation}
\cventry{Almost ready}{On derived D-modules}{Find \href{https://carlobuccisano.github.io/article}{here} the updated version.}{}{}{}
%----------------------------------------------------------------------------------------
%	ORGANIZER SECTION
%----------------------------------------------------------------------------------------
\section{Co-organized Conferences}
\cventry{May 2022}{GAeL XXIX}{Orsay, France}{}{}{}


%----------------------------------------------------------------------------------------
%   TALKS SECTION
%----------------------------------------------------------------------------------------

\section{Selected Talks}

\cventry{August 2024}{Derived D-Modules}{Poster session in Young Topologists Meeting, 2024}{}{}{}
\cventry{April 2024}{Grothendieck construction and quotient stacks}{in GIT reading group}{}{}{}
\cventry{June 2023}{Factorization homology and chiral Koszul duality}{in online factorisation seminar}{}{}{}
\cventry{Oct 2022}{Introduction to Derived Algebraic Geometry}{in \textit{Donaldson-Thomas invariants in Aussois}}{}{}{}
\cventry{Jan 2022}{Introduction to Poisson Geometry}{in Symplectic Duality reading group}{}{}{}


%----------------------------------------------------------------------------------------
%	CONFERENCE SECTION
%----------------------------------------------------------------------------------------
%\begin{comment}
\section{Attended Conferences}

\cventry{August 2024}{Young Topologists Meeting}{University of M{\"u}nster, Germany}{}{}{}
\cventry{May 2024}{Higher Algebra, Geometry, and Topology}{CIRM Marseille, France}{}{}{}
\cventry{October 2023}{Categories and stacks in algebraic geometry and algebraic topology CATS 7}{CIRM Marseille, France}{}{}{}
\cventry{July 2023}{Young Topologists Meeting 2023}{EPFL Lausanne, Switzerland}{}{}{}
\cventry{Apr-July 2023}{Higher Structures in Geometry and Mathematical Physics}{IHP Paris, France}{}{}{}
\cventry{Oct 2022}{Donaldson-Thomas invariants in Aussois}{Aussois, France}{}{}{}
%\cventry{June 2022}{Nouveau séminaire ALPE}{IMT Toulouse, France}{}{}{}
\cventry{June 2022}{CATS-60 Conférence A Toulouse pour Simpson}{IMT Toulouse, France}{}{}{}
\cventry{June 2022}{Workshop on Derived Geometry}{CRM Barcelona, Spain}{}{}{}
\cventry{May 2022}{Spring School on Field Theories and Algebraic Topology}{Driebergen, Netherlands}{}{}{}
\cventry{March 2022}{Nouveau séminaire ALPE}{IMAG Montpellier, France}{}{}{}
\cventry{Sept 2021}{Summer School - Irregular Riemann-Hilbert Correspondence}{Aussois, France}{}{}{}
%\end{comment}



%----------------------------------------------------------------------------------------
%	TEACHING SECTION
%----------------------------------------------------------------------------------------

\section{Teaching experience}

\cventry{Fall 2024}{Analyse 2 (L2)}{Université Paris-Saclay}{exercise sessions}{}{}
\cventry{Fall 2024}{Algèbre linéaire (L1 et L3)}{Université Paris-Saclay}{exercise sessions}{}{}

\cventry{Fall 2022}{Structures Algébriques (L2)}{Université Paris-Saclay}{exercise sessions}{}{}

\cventry{Fall 2022}{Calculus (L1)}{Université Paris-Saclay}{exercise sessions}{}{}

\cventry{Spring 2022}{Maths 3 - Analyse 1 (L1)}{Université Paris-Saclay}{exercise sessions}{}{}

\cventry{Spring 2022}{Analyse et Géométrie (L2)}{Université Paris-Saclay}{exercise sessions}{}{}
%----------------------------------------------------------------------------------------
%	WORK EXPERIENCE SECTION
%----------------------------------------------------------------------------------------
\section{Work experience}


\cventry{July 2015}{Computer Programmer}{\textsc{IBM}}{Hursley, UK}{}{}

%----------------------------------------------------------------------------------------
%	RESPONSABILITIES SECTION
%----------------------------------------------------------------------------------------
\section{Responsabilities}
\cvitem{2021--2022}{Representative for PhD students at the Doctoral School of Mathematics Hadamard, and at Orsay Maths Department.}

%----------------------------------------------------------------------------------------
%	AWARDS SECTION
%----------------------------------------------------------------------------------------
\section{Awards}
\cvitem{2016--2022}{School of Honors Scholarship by Scuola Galileiana di Studi Superiori, Padova.} 
\cvitem{2020}{Scholarship Sophie Germain by Fondation Mathématique Jacques Hadamard.}
\cvitem{2013--2015}{Italian Olympiad in Informatics (1 bronze and 2 gold medals). }
\cvitem{2015}{Italian Mathematical Olympiads, silver medal.}
\cvitem{2014}{Prize for the study of Maths and Computer Science, Banca d'Italia.}


%----------------------------------------------------------------------------------------
%	COMPUTER SKILLS SECTION
%----------------------------------------------------------------------------------------

\section{Computer skills}

\cvitem{Intermediate}{\texttt{Java}, \texttt{git}, OpenOffice, Windows}
\cvitem{Advanced}{\texttt{C++}, \texttt{Python}, \LaTeX, Linux}

%----------------------------------------------------------------------------------------
%	LANGUAGES SECTION
%----------------------------------------------------------------------------------------

\section{Languages}
\cvitemwithcomment{Italian}{Native speaker}{}
\cvitemwithcomment{English}{Advanced, IELTS Academic Band 8.0}{Conversationally fluent}
\cvitemwithcomment{French}{Advanced}{Conversationally fluent}
\cvitemwithcomment{Spanish}{Basic}{}

\end{document}
