%%%%%%%%%%%%%%%%%%%%%%%%%%%%%%%%%%%%%%%%%
% "ModernCV" CV and Cover Letter
% LaTeX Template
% Version 1.3 (29/10/16)
%
% This template has been downloaded from:
% http://www.LaTeXTemplates.com
%
% Original author:
% Xavier Danaux (xdanaux@gmail.com) with modifications by:
% Vel (vel@latextemplates.com)
%
% License:
% CC BY-NC-SA 3.0 (http://creativecommons.org/licenses/by-nc-sa/3.0/)
%
% Important note:
% This template requires the moderncv.cls and .sty files to be in the same 
% directory as this .tex file. These files provide the resume style and themes 
% used for structuring the document.
%
%%%%%%%%%%%%%%%%%%%%%%%%%%%%%%%%%%%%%%%%%

%----------------------------------------------------------------------------------------
%	PACKAGES AND OTHER DOCUMENT CONFIGURATIONS
%----------------------------------------------------------------------------------------

\documentclass[10pt,a4paper,sans]{moderncv} % Font sizes: 10, 11, or 12; paper sizes: a4paper, letterpaper, a5paper, legalpaper, executivepaper or landscape; font families: sans or roman

\moderncvstyle{classic} % CV theme - options include: 'casual' (default), 'classic', 'oldstyle' and 'banking'
\moderncvcolor{grey} % CV color - options include: 'blue' (default), 'orange', 'green', 'red', 'purple', 'grey' and 'black'

\usepackage[scale=0.9]{geometry} % Reduce document margins
\setlength{\makecvheadnamewidth}{8cm} % For the 'classic' style, uncomment to adjust the width of the space allocated to your name

%----------------------------------------------------------------------------------------
%	NAME AND CONTACT INFORMATION SECTION
%----------------------------------------------------------------------------------------


\name{Carlo}{Buccisano}

% All information in this block is optional, comment out any lines you don't need
\title{Curriculum Vitae}
\address{Université Paris-Saclay}{Département de Mathématiques d'Orsay}{Bat 307, rue Michel Magat, Orsay, France}
\email{carlo.buccisano@universite-paris-saclay.fr}


%----------------------------------------------------------------------------------------

\definecolor{linkcolor}{rgb}{0.0, 0.44, 1.0}

\begin{document}

\makecvtitle % Print the CV title

%----------------------------------------------------------------------------------------
%	EDUCATION SECTION
%----------------------------------------------------------------------------------------

\section{Education}
\cventry{2021--present}{PhD student in Mathematics}{Université Paris-Saclay}{Departement de Mathématiques d'Orsay}{}{Advisors: Benjamin Hennion and Emanuele Macrì.\newline Title: \textit{Chiral differential operators in higher dimension}.}
\cventry{2020--2021}{Second year master student in Mathematics}{Université Paris-Saclay}{M2 Arithmétique, Analyse,  Géométrie}{}{Weighted grade average: 18.675/20.\newline Mention: \textit{Très Bien} (maximum honours).}
\cventry{2019--2020}{First year master student in Mathematics}{Università degli Studi di Padova}{}{}{Weighted grade average: 30/30. }
\cventry{2016--2019}{Bachelor in Mathematics}{Università degli Studi di Padova}{}{}
	{Weighted grade average: 29.965/30.\newline Mention: \textit{110/110 cum laude} (maximum honours).}

%----------------------------------------------------------------------------------------
%	CONFERENCE SECTION
%----------------------------------------------------------------------------------------
\section{Attended conferences}

\cventry{Sept 2021}{Summer School - Irregular Riemann-Hilbert Correspondence}{Aussois, France}{}{}{}
\cventry{March 2022}{Nouveau séminaire ALPE}{Montpellier, France}{}{}{}
%\cventry{May 2022}{Spring School on Field Theories and Algebraic Topology}{Driebergen, Netherlands}{}{}{}
%\cventry{June 2022}{CATS-60 Conférence A Toulouse pour Simpson}{Toulouse, France}{}{}{}
%\cventry{Oct 2022}{Donaldson-Thomas invariants in Aussois}{Aussois, France}{}{}{}


%----------------------------------------------------------------------------------------
%	ORGANIZER SECTION
%----------------------------------------------------------------------------------------
\section{Co-organized conferences}
\cventry{May 2022}{GAeL XXIX}{Orsay, France}{}{}{}


%----------------------------------------------------------------------------------------
%	TEACHING SECTION
%----------------------------------------------------------------------------------------

\section{Teaching experience}

\cventry{Spring 2022}{Maths 3 - Analyse 1}{Université Paris-Saclay}{}{}{Exercise session for the course Maths 3- - Analyse 1 for first year students of double degree mathematics-computer science.}

\cventry{Spring 2022}{Analyse et Géométrie}{Université Paris-Saclay}{}{}{Exercise session for the course Analyse et Géométrie for second year students of double degree mathematics-physics.}
%----------------------------------------------------------------------------------------
%	WORK EXPERIENCE SECTION
%----------------------------------------------------------------------------------------
\section{Work experience}


\cventry{July 2015}{Computer Programmer}{IBM}{Hursley, UK}{}{}

%----------------------------------------------------------------------------------------
%	AWARDS SECTION
%----------------------------------------------------------------------------------------
\section{Awards}


\cvitem{2016--2022}{School of Honors Scholarship by Scuola Galileiana di Studi Superiori,Padova.} 
\cvitem{2020}{Scholarship Sophie Germain by the Fondation Mathématique Jacques Hadamard.}
\cvitem{2013--2015}{Italian Olympiad in Informatics (1 bronze and 2 gold medals). }
\cvitem{2015}{Italian Mathematical Olympiads, silver medal.}
\cvitem{2014}{Prize for the study of Maths and Computer Science, Banca d'Italia.}

\section{Languages}
\cvitemwithcomment{Italian}{Native speaker}{}
\cvitemwithcomment{English}{Advanced, IELTS Academic Band 8.0}{Conversationally fluent}
\cvitemwithcomment{French}{Intermediate}{Self assessed B2}

\end{document}